This thesis addresses turbulence and transport in the plasma edge of tokamaks, a region that extends around the separatrix up to the wall, and separates closed and open magnetic field lines. Confined magnetic fusion in such devices is a constantly developing technology that aims to generate energy from the fusion of light atoms, as it occurs in stars. Understanding and characterizing the physical phenomena in this plasma region is crucial to guarantee both the safety and performance of future reactors like ITER. These phenomena govern both confinement and exhaust properties. In this work, a new electromagnetic model was introduced to the state-of-the-art SOLEDGE3X code, in which magnetic induction and magnetic perturbations affect the dynamics of drift-wave turbulence. This lays the foundations for new, more relevant turbulence simulations to study larger machines and plasmas at higher temperatures, which should contribute to a better understanding of the theory of edge physics and to the optimisation of future reactor concepts such as ITER. \\

In Chapter \ref{chap:TokamakConcept}, the fundamental physics of particles in a magnetized plasma were introduced and applied to the specific configuration of a tokamak. Special attention was given to the magnetohydrodynamic equilibrium between thermodynamic and magnetic pressures, resistive and diffusive processes resulting from particle collisions, and the specific physics of the sheath at the boundary between the plasma and the reactor wall. In Chapter \ref{chap:PlasmaSimulations}, we explained how the plasma can be described mathematically such that supercomputers can solve relevant physics. Starting from a direct particle description, including (gyro-)kinetic approaches, we derived the magnetohydrodynamic (MHD) description of plasmas that studies the evolution of mean fields similar to methods from computational fluid dynamics. This model can be further extended to drift-reduced fluid models, which are particularly well-suited to studying drift-wave turbulence in the plasma edge and the Scrape-Off Layer (SOL). We then discussed how electromagnetic effects in Ohm's law modify the non-adiabatic electron response to perturbations. They have a non-negligible impact on the turbulent dynamics in this region, even when the pressure ratio $\beta$ has very low values. \\

This leads us to Chapter \ref{chap:SOLEDGE3X_framework}, where we presented the new electromagnetic model in SOLEDGE3X that constitutes the main contribution of this work. It can be broken down into three components. First, magnetic induction appears in the parallel electric field $E_\parallel$ through the temporal change of the parallel magnetic vector potential $\partial_t A_\parallel$. The unknown $A_\parallel$ is computed from the parallel current density $j_\parallel$ using Ampère's law. These new fields drive fluctuations $\tilde{B}$ in the magnetic field, commonly called flutter, which form the second component of the new model. The fluctuations are assumed to be small compared to the equilibrium field and are applied at first order. They reorient parallel transport channels and modify parallel gradients in Ohm's law or parallel heat conduction. Finally, taking a nonzero electron mass in Ohm's law adds electron inertia to the non-adiabatic response. This effect is necessary to constrain the Alfvén speed to physical values that might occur at steep perpendicular gradients and substitutes electric resistivity in hot plasmas. In Chapter \ref{chap:Implementation}, more technical details of the new model were explained. Two new fields had to be added: the parallel projection of the magnetic vector potential $A_\parallel$ and the parallel current density $j_\parallel$, required to solve electron inertia effects. In the flux-surface aligned meshing of SOLEDGE3X, they are defined on a poloidally and toroidally staggered grid to benefit from first-order parallel derivatives, requiring a whole set of new discrete operators. In the implicit-explicit time discretization, collisional processes are historically solved implicitly, which includes the vorticity equation on the electric potential $\Phi$. To avoid additional restrictions on the CFL condition, the new Alfvénic and electron inertial terms are solved implicitly in a large coupled 3D system with the vorticity equation. Due to the complex structure of the system, magnetic induction deteriorates the matrix condition; however, taking a finite electron mass reduces the anisotropy between the perpendicular and parallel Laplacians on $\Phi$ and hence improves the condition. Flutter adds a severe layer of complexity to the entire model, as magnetic fluctuations introduce a radial component to the magnetic field. It required extensions of all parallel operators and had a particularly significant impact on the implicit viscosity and parallel heat conduction problems, as the coupling between magnetic flux surfaces imposed new 3D operators. The implementation was verified and validated in Chapter \ref{chap:VV}. Using the method of manufactured solutions, the new electromagnetic vorticity equation and the 3D heat diffusion with flutter were checked for correctness and second-order spatial convergence. Integration tests on slab geometries allowed recovery of frequencies from linear theory, especially the transition from shear Alfvén to thermal electron waves as the perpendicular wavenumber increases. \\

The final part is dedicated to applications in turbulent simulations. In Chapter \ref{chap:analSimulations}, we explored electromagnetic effects on simple limited geometries. In a long toroidal slab with a pure toroidal magnetic field and a limiter at each end, we observed that magnetic induction enhances radial blob propagation at a certain distance from the sheath. With a sharp-edged heat and particle source, magnetic induction promotes the generation and expansion of turbulent structures, while flutter has a clear stabilizing effect on them. The next set of tests were performed on a circular geometry with a flat limiter on the low-field side. A scan on the electron mass $m_e$ and the electromagnetic parameter $\beta$ showed that both effects accelerate the initial growth of instabilities as one would expect from linear theory. Adding flutter to the system reduces the growth rate again, and this is also an expected effect. Complex diverted geometries with X-points were addressed in Chapter \ref{chap:TCV}, with a comparison between electrostatic, electron inertia, inductive, and fully electromagnetic models based on the TCV-X21 benchmark. In this L-mode plasma, it was found that magnetic induction alone significantly increases radial particle and heat transport across the separatrix, such that the temperature can barely build up and quickly reaches a quasi-steady state on energy. The stabilizing effect of flutter is quite present here, as the energy confinement returns, or even surpasses, electrostatic levels. Nonetheless, ExB drifts always dominate the cross-field transport and are much stronger than electromagnetic energy advection or redirection of the parallel heat flux. The last Chapter \ref{chap:RippleMagnetic} shows applications of the new model beyond the realm of turbulent transport and electromagnetic effects. Perturbations to the magnetic equilibrium can also be imposed externally, allowing us to run non-axisymmetric magnetic configurations, as long as the perturbations remain small compared to the axisymmetric average. This is particularly useful for studying magnetic ripple, small toroidal and poloidal perturbations of $B$ due to the toroidal locality of poloidal field coils, which was observed to be particularly strong on the WEST tokamak in Cadarache. Preliminary results of a simulation in transport mode exhibited a snake-skin pattern on the divertor plates with alternating peaks in heat deposition on the inner and outer targets, that were previously only observed on infrared imagery. \\


\vspace{3cm}

One key issue for self-consistent turbulent simulations is the very long execution time to reach convergence, and the electromagnetic model further worsens code performance. To obtain stable and significant mid-plane profiles, one must wait at least for a particle confinement time, which can reach up to 100ms for TCV and be much higher for larger machines. The simulations presented in this thesis were still far from convergence, even though the mesh used was still very coarse. While investigating drift-wave turbulence is important for understanding edge plasma behavior and cross-field transport, it is of limited use for global simulations. The current setup could benefit from improved initial conditions to reduce computational overhead. One option would be to start from experimental profiles that, while lacking fine-grained density and temperature distributions or toroidal fluctuations, could considerably accelerate the time to reach a quasi-steady state in an already converged environment. Our simulations showed that turbulent dynamics establish and respond quickly to changes in global profiles. This could allow for studying drift-wave turbulence in specific scenarios, such as L- or H-mode on a given tokamak or during an ELM event. The main drawback of using experimental profiles is that the model must perfectly match reality, otherwise slightly different profiles could establish, and the problem of waiting a full confinement time re-emerges. Alternatively, coupling turbulent and transport modes could be considered, where mean profiles evolve in a fast 2D simulation, with diffusion coefficients for cross-field transport regularly updated from 3D turbulent simulations restarted at specific intervals.

Regarding the electromagnetic model, the new terms have a considerable impact on turbulent dynamics and appear necessary for an accurate modeling of both interchange and drift-wave instabilities. Electron inertia is already beneficial for existing electrostatic scenarios, offering numerical advantages and regulating plasma dynamics in regions of low resistivity. However, it is not advisable to include magnetic induction without flutter or electron inertia; the full electromagnetic model should always be used. Preliminary results suggest that flutter stabilizes turbulent dynamics, but its precise role in forming or maintaining transport barriers, particularly in higher $\beta$ scenarios, remains unclear. Further analysis of flutter in complex configurations with H-mode plasmas could open the door to exploring turbulence in future machines like ITER or DEMO. However, in the current implementation, simulations with flutter come at a significant numerical cost, as it introduces a radial component to the magnetic field, leading to flux surface coupling in the aligned meshing. This affects all implicit solvers involving (now fluctuating) parallel gradients, including those for viscosity, parallel heat conduction, and vorticity. A potential solution could involve an explicit-implicit splitting between the equilibrium and fluctuating field components. Since flutter is added at first order and is required to be much smaller than the equilibrium field, it may not impose strong restrictions on the CFL condition in explicit time-stepping. The hybrid approach would preserve model accuracy while making it more suitable for longer simulated times, higher resolutions, or larger tokamaks.

Building on the electromagnetic framework, future developments of SOLEDGE3X could target simulations of edge localized modes (ELMs) and their suppression with resonant magnetic perturbations (RMPs). The main challenge in capturing these instabilities is the separation of large-scale dynamics over long timescales from the smaller-scale drift-wave turbulence, which may require calculating the flutter field without solving for drifts. Currently, the model does not solve for the necessary bootstrap current, which arises from radial pressure gradients in conjunction with trapped electrons. It is also unclear whether the perturbative approach $\mathbf{B} = \mathbf{B}_{eq} + \tilde{\mathbf{B}}$ is sufficient to grow large magnetic islands that could trigger tearing modes. The ripple simulations presented in this work were a first step toward modeling non-axisymmetric magnetic fields and open the door to more ambitious configurations. For example, ergodic divertors are designed to break magnetic field lines into a chaotic state, spreading plasma heat fluxes over a wider area on the divertor plates.







