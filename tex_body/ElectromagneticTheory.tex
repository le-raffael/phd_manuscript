
%\section{Electromagnetism in Other Software Projects}
%The fluid model is a somewhat popular approach to simulate transport phenomena of edge plasma and is followed by various research groups in the fusion community. Among the major actors appear the..., ..., ... . This section gives more detailed insights into the Bout++ and GRILLIX software projects, as they are both relevant for the further research done within this thesis.
%\subsection{Bout++}
%\subsection{GBS}
%\subsection{GRILLIX}
%The Max-Planck institute for plasma physics in Garching hosts the GRILLIX software project \cite{GrillixGeneralPaper} for turbulent plasma transport in the edge region on flexible 3D geometries. It distinguishes by the use of a flux-coordinate independent (FCI) approach  \cite{GrillixFCIMethod} which uses a cylindrical grid to span the tokamak. Under the assumption of a strong toroidal field it is acceptable to discretize perpendicular operators only on the Cartesian poloidal planes with typical stencils. Because the discretization is not aligned with fields or fluxes, issues with singularities along the separatrix or at the X-point are easily avoided. Parallel operators are calculated by field line tracing and subsequent interpolation which allows to use only very few poloidal planes together with a high perpendicular resolution. \\
%The Karniadakis method \cite{KarniadakisScheme} for the time advancement offers an appropriate framework to treat implicitly all terms that depend on the stiff parallel current and explicitly all other terms. At each timestep $t$, t    he following system of equation has to be solved for the density logarithm $\theta_n=\ln(n)$, the parallel ion velocity $u_\parallel$, the parallel current density $j_\parallel$ and the electric potential $\Phi$:
%\begin{equation}
%	\begin{pmatrix}
%		1 & 0 & -\frac{6}{11 n^t}\delta t \grad_\parallel  & 0 \\
%		0 & 1 & 0 & 0 \\
%		-\sigma \grad_\parallel & 0 & 1 & \sigma \grad_\parallel \\
%		0 & 0 & -\frac{6}{11}\delta t \grad_\parallel & \frac{1}{B^2}\Delta_\perp 
%	\end{pmatrix} \begin{pmatrix}
%		\theta_n^t \\ u_\parallel^t \\ j_\parallel^t \\ \Phi^t
%	\end{pmatrix} = \begin{pmatrix}
%		S_\theta \\ S_{u_\parallel} \\ 0 \\ S_\Omega
%	\end{pmatrix} \label{eq:GrillixElectrostaticImplicitSystem}
%\end{equation}
%All operators in the matrix stand for their discrete stencils, the non-linear $n^t$ is extrapolated from the three previous timesteps and the right-hand side terms $S_*$ contain the Karniadakis scheme and all explicitly solved fields. As a note, the factor $6/11$ originates in the time-stepping Karniadakis scheme. \\
%In S3X the field $j_\parallel$ in never directly computed but it is yet very present in the solved system; The gradient of the electric potential appears in the parallel Laplacian of the vorticity equation Ohm's RHS of the vorticity system with its electronic pressure and temperature gradients. The challenge of a combined parallel and perpendicular diffusion on $\Phi$, origin of the high anisotropy in S3X and main subject of this thesis, does thus not exist in GRILLIX. Further the ion density and velocity are solved explicitly in our code, so the only implicitly solved field is the electric potential $\Phi$ with its high anisotropy. \\
%In 2019, the system of equation in GRILLIX was extended by the 


