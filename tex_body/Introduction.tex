\chapter{Introduction}
\label{chap:Intro}
The sun is the primary source of energy for Earth, essential for photosynthesis in plants, which forms the basis of most food chains, and for driving the weather and climate systems that shape our environment. Its consistent radiation supports all life forms, regulates global temperatures, and influences fundamental ecological and biological processes that are vital for the Earth's diverse ecosystems. From the very outset of human life, the sun has been a subject of profound admiration, occupying a central role in various religious beliefs and was often synonym of an incomparably vast and potent source of energy. It was not until the beginning of the twentieth century that progress in particle physics allowed to unravel the secret of solar energy: nuclear fusion. It is the physical process where two light atomic nuclei merge to form a heavier nucleus, releasing significant energy as a result of mass-to-energy conversion. \newline 
The dream of achieving nuclear fusion in a laboratory to produce energy emerged shortly thereafter. In today's climate crisis, nuclear fusion is even more appealing because it does not emit carbon emissions, does not present the risk of a catastrophic meltdown and its fuel, hydrogen, is readily available. Since replicating the sun's core conditions on Earth, particularly the immense pressure, is not feasible, alternative approaches were searched. A look at the reaction cross-section of various pairs of light atoms shows that deuterium-tritium (D-T) fusion has the highest likelihood at the lowest temperature. These two hydrogen isotopes are hence the most favorable candidates for fusion; deuterium is naturally abundant, but tritium, which is radioactive and has a relatively short half-life, must be artificially produced. \newline

At such elevated temperatures, the binding energy is insufficient to maintain the cohesion of electrons and atomic nuclei, resulting in the formation of a state of matter known as plasma. Fundamentally, plasma is an ionized gas composed of positively charged nuclei and negatively charged electrons, which interact electromagnetically. \newline

Lawson's criterion\cite{Lawson1957} estimates the necessary plasma conditions to reach the break-even point, when fusion power exceeds heating and conduction losses. For D-T fusion, the triple product of density $n$, temperature $T$ and confinement time $\tau_E$ must exceed:

\begin{equation}
	\label{eq:LawsonCriterionDT}
	nT\tau_E > 10^{-21} \text{keV m}^{-3}\text{s}
\end{equation}

The reaction cross-section determines an optimal temperature of approximately 15-40 keV  (~150 millions °C) for fusion reactions, leading fusion reactor designs to focus on maximizing either of the two remaining parameters: density or confinement time. Inertial Confinement Fusion (ICF) seeks to compress dense fuel pellets for an extremely brief duration using high-powered lasers. Conversely, Magnetic Confinement Fusion (MCF) utilizes strong magnetic fields to sustain stable plasmas at relatively low densities. Both approaches have conducted promising experiments close to the break-even point. Within MCF, there are two primary designs: tokamaks, which use a toroidal chamber with an axisymmetric magnetic field, and stellarators, which use a twisted magnetic configuration to improve plasma confinement. \newline

\begin{figure}[H]
	\centering
	\includegraphics[width=0.62\textwidth]{schemes/fusion-reactivities.png}
	\caption{Fusion reaction cross-sections for the most promising pairs of light elements over plasma temperature.}
	\label{fig:Intro_fusionCrossSections}
\end{figure}

%The name "tokamak" or "токамак" itself is a Russian acronym of "тороидальная камера с магнитными катушками", which translates to "toroidal chamber with magnetic coils" and directly refers to its magnetic configuration. \\

For MCF \\
Plasma heating \\ 
- Ohmic heating \\ 
- ICRH, ECRH, NBI \\

fusion gain Q as essential metric. \\

show diagram with all machines, scaling laws for fusion performance \\

The largest fusion experiment ITER, is currently under construction in southern France by an international collaboration of seven member parties. At its full operation it is expected to achieve ignition, a state where the fusion reaction emits sufficiently radiation to maintain the plasma conditions. It requires a heat exhausts ten times higher than the break-even point and is a critical milestone for the development of future commercial fusion reactors. \\

Introduce necessity for (electromagnetic) turbulent simulations (estimate cross-field transport, power exhaust,ELMs...)  \\

Currently turbulent SOLEDGE3X limited to L-modes and small machines. Numerical issues and limited model for larger machines. Limitations due to the anistropy between the parallel (resistive) and perpendicular (from the time evolution of the vorticity) Laplacian on the potential. Problematic if the resistivity gets very small. Even in the electrostatic collisional regime found in the plasma edge, electron inertial and electromagnetic effects play a substantial role. Especially electron inertia replaces resistivity as it approaches zero. \\

The current implementation of SOLEDGE3X is primarily limited to simulating L-mode plasmas and smaller tokamaks due to various numerical challenges and the limited applicability of its models to larger machines. A significant limitation arises from the anisotropy between the parallel (resistive) and perpendicular (vorticity-driven) Laplacians acting on the electric potential. In the existing electromagnetic model, the vorticity equation is solved implicitly and as the resistivity approaches zero, the condition number of the matrix deteriorates considerably. Even in the electrostatic, collisional regime typically found in edge plasma, electron inertia and electromagnetic effects play a crucial role, and notably the finite electron mass in Ohm's law acts as a lower bound for the resistivity. \\

As the plasma approaches the LH transition, electromagnetic effects become increasingly important. The H-mode is characterized by a suppression of cross-field transport due to "ExB" drifts, which gets partly replaced by electromagnetic transport. Significant magnetic reconnection processes lead to an important transport of plasma particles from the hot core to the cold edge, with radial fluxes still below "ExB" advection, but non-negligible to understand the physical plasma behavior. \\

This thesis is dedicated to the implementation of an electromagnetic model within SOLEDGE3X, which includes magnetic induction in the parallel electric field, perturbations of the magnetic equilibrium (flutter), and a finite electron mass in Ohm's law. This development pursues several goals: first, it improves the accuracy of the physical model; second, it enhances the numerical robustness by mitigating the poor matrix conditioning associated with low resistivity; and third, it establishes a foundation for self-consistent turbulent simulations that are relevant to larger machines and higher-power scenarios.