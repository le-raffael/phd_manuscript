\addchap{Abstract and keywords}
\label{chap:abstract}

In the tokamak edge, steep gradients and magnetic curvature generate macroscopic turbulent structures that transport plasma particles from the hot core, where fusion occurs at around 10 keV, to the much colder Scrape-Off-Layer (SOL), where magnetic field lines intersect the physical wall. Turbulence reduces plasma confinement and determines the area where strong heat fluxes impact the divertor. The drift-reduced fluid code SOLEDGE3X, developed at CEA Cadarache, has proven effective to simulate electrostatic resistive drift-wave turbulence in L-mode plasmas. However, recent research has shown that electromagnetic effects have a non-negligible impact on drift-wave dynamics even at low $\beta$, especially in scenarios with higher temperatures. \\

This thesis introduces a new electromagnetic model in SOLEDGE3X for the vorticity equation, incorporating magnetic induction, electromagnetic flutter, and electron inertia. Magnetic induction accounts for the time variation of the parallel magnetic vector potential $A_\parallel$ in the definition of the parallel electric field, and $A_\parallel$ is known from the parallel current density $j_\parallel$ via Ampère's law. Fluctuations in the magnetic field, called flutter, are added at first order and assumed small compared to the equilibrium field. Electron inertia, represented by a finite electron mass in Ohm's law, is necessary to constrain shear Alfvén wave speeds to physical values. The new fields $A_\parallel$ and $j_\parallel$ are integrated into the flux-surface aligned FVM framework on a poloidally and toroidally staggered grid. Flutter affects parallel transport equations and gradients in Ohm's law, and its implementation required special care to account for the new radial component of the parallel direction $\textbf{b}$. To handle timesteps larger than Alfvénic, electron thermal or electron-ion collision times, the corresponding inductive, inertial, and resistive effects are solved implicitly in a coupled 3D system on the potentials $\Phi$ and $A_\parallel$. The model was verified with manufactured solutions and validated on a linear slab case, that could exhibit the expected transition from Alfvén to thermal electron waves as the perpendicular wavenumber increased. \\ 

Flutter contributes minimally to cross-field transport but impacts the non-adiabatic potential response to density fluctuations in Ohm's law. Simulations in slab, circular (limited), and X-point (diverted) geometries consistently show that electron inertia and magnetic induction destabilize drift-wave turbulence, while flutter stabilizes it, both in the linear and nonlinear phases. On open field lines, magnetic induction reduces blob sensitivity to sheath effects, promoting further turbulence spreading in the SOL. Preliminary results from a power scan on a TCV scenario indicate the formation of a transport barrier at the separatrix for the higher-power cases. Numerically, electron inertia significantly improves the condition number of the vorticity system, especially in hot plasmas with low resistivity, providing a factor four speed-up even in electrostatic scenarios. However, adding flutter degrades code performance, as it requires solving implicit 3D systems for viscosity and heat diffusion problems that were previously treated as uncoupled 2D systems on each flux surface. As an extension to the work, perturbations to the magnetic equilibrium can also be imposed externally, and it was demonstrated in a transport mode simulation studying heat deposition in a non-axisymmetric magnetic configuration with ripple on WEST.


\vspace{0.5cm}
Keywords: tokamak, edge plasma, turbulent simulations, electromagnetism, SOLEDGE3X
