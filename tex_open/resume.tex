\addchap{Résumé et mots clés}
\label{chap:resume}

\selectlanguage{french}

Dans le bord du tokamak, les gradients abrupts et la courbure magnétique génèrent des structures turbulentes de grande échelle qui transportent les particules de plasma du cœur chaud, où la fusion se produit à environ 10 keV, vers la couche limite (SOL) beaucoup plus froide, où les lignes de champ magnétique croisent la paroi. La turbulence réduit le confinement du plasma et détermine la zone où de forts flux de chaleur impactent le divertor. Le code fluide SOLEDGE3X, développé par le CEA/IRFM en collaboration avec Aix-Marseille Université, s'est avéré efficace pour simuler la turbulence électrostatique résistive des ondes de dérive dans des géométries de tokamak réalistes. Cependant, des résultats expérimentaux et numériques ont montré que les effets électromagnétiques ont un impact significatif sur la dynamique des ondes de dérive, et donc sur la turbulence de bord. \\

Cette thèse introduit un modèle électromagnétique dans SOLEDGE3X, avec trois composantes : l'induction magnétique, le flutter électromagnétique et l'inertie des électrons. L'induction magnétique tient compte de la variation temporelle du potentiel vecteur magnétique parallèle $A_\parallel$ dans la définition du champ électrique parallèle, et $A_\parallel$ est relié à la densité de courant parallèle $j_\parallel$ par la loi d'Ampère. Les fluctuations du champ magnétique, appelées flutter, sont ajoutées au premier ordre et supposées faibles par rapport au champ d'équilibre. L'inertie des électrons, apparaissant avec une masse d'électrons non nulle dans la loi d'Ohm, est nécessaire pour contraindre les vitesses des ondes d'Alfvén à des valeurs physiques. Les nouveaux champs $A_\parallel$ et $j_\parallel$ sont intégrés dans le maillage aligné aux surfaces de flux sur une grille décalée poloïdalement et toroïdalement. Le flutter affecte les équations de transport parallèle et les gradients dans la loi d'Ohm. Sa mise en œuvre a requis une attention particulière pour tenir compte de la nouvelle composante radiale de la direction parallèle $\textbf{b}$. Pour permettre des pas de temps plus importants que les temps d'Alfvén, de transit électronique ou de collision électron-ion, les effets inductifs, inertiels et résistifs sont résolus implicitement dans un système 3D couplé sur les potentiels $\Phi$ et $A_\parallel$. Le modèle a été vérifié au moyen de solutions analytiques et validé sur un cas linéaire, montrant la transition d'ondes d'Alfvén à des ondes électroniques thermiques. \\

Le flutter contribue peu au transport radial, mais influence la réponse non adiabatique du potentiel aux fluctuations de densité. Les premières simulations dans des géométries slab, circulaires (limitées) et à point X (à divertor) montrent de manière consistante que l'inertie des électrons et l'induction magnétique déstabilisent la turbulence des ondes de dérive, tandis que le flutter la stabilise, à la fois dans les phases linéaires et non linéaires. Sur les lignes de champ ouvertes, l'induction magnétique réduit la sensibilité des structures turbulentes aux effets de gaine, favorisant la propagation de la turbulence dans la SOL. Sur le plan numérique, l'inertie des électrons améliore considérablement le conditionnement de la matrice de vorticité, en particulier dans les plasmas chauds à faible résistivité, accélérant la résolution d'un facteur quatre. Cependant, l'ajout du flutter dégrade la performance du code, car il nécessite la résolution de systèmes 3D implicites pour les problèmes de viscosité et de diffusion thermique, auparavant traités comme des systèmes 2D découplés sur chaque surface de flux. Dans le prolongement de ce travail, des perturbations de l'équilibre magnétique ont été imposées de manière externe dans une simulation en mode transport portant sur le dépôt de chaleur dans une configuration magnétique non axisymétrique avec ripple sur WEST.




\vspace{0.5cm}
Mots clés: tokamak, plasma de bord, simulations turbulentes, électromagnétisme, SOLEDGE3X


\selectlanguage{english}										

