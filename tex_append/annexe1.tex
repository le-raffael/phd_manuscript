\chapter{Curvilinear metric operators}
\label{app:CurvilinearMetric}

\section{Centered operators}
\label{sec:app_centeredOperators}

\subsection{Differentiation}
The gradient of a scalar field $S(u^\psi, u^\theta, u^\varphi)$ is calculated in terms of the reciprocal basis: 
\begin{align}
	\left(\grad{S}\right)_i = \pdv{S}{u^i}\vec{e}^i \label{eq:MetricGradient}
\end{align}
In the transport equations, we need to calculate gradients in parallel direction of the field which allows us to express $\vec{\grad}_\parallel$ in terms of the contravariant vector of the unit magnetic field from \autoref{eq:MetricContravariantUnitB}.
\begin{align}
	\grad_\parallel S = \pdv{S}{u^i}\vec{b}\cdot\vec{e}^i = \pdv{S}{u^\theta}b^\theta + \pdv{S}{u^\varphi}b^\varphi \label{eq:MetricParallelGradient}
\end{align}
The terms in $\psi$ are omitted in the above formula since the equilibrium magnetic flux surfaces are located on the $\theta$-$\varphi$ coordinate surface and the vector $\vec{b}$ has therefore a zero value in its radial component. Note that this gradient is a scalar as it always points in $\vec{b}$ direction. \\
Similarly, a perpendicular gradient can be defined as: 
\begin{equation*}
	\vec{\grad}_\perp S = \vec{\grad} S - \left(\grad_\parallel S\right)\vec{b}
\end{equation*}
In terms of metric coefficients, it translates to: 
\begin{align}
	\label{eq:MetricPerpendicularGradient}
	\left(\vec{\grad}_\perp S\right)^i =& g^{ij}\left(\grad{S}\right)_j - \left(\grad_\parallel S\right)b^i \nonumber \\
	=& g^{ij}\pdv{S}{u^j} - \left(\pdv{S}{u^\theta}b^\theta + \pdv{S}{u^\varphi}b^\varphi\right)b^i
\end{align}
We have to keep in mind that the above expression leads to the contravariant components of the perpendicular gradient whereas the general expression for the gradient in \autoref{eq:MetricGradient} gives its covariant components. \\

Next, the divergence of a vector $\vec{v}$ is calculated by: 
\begin{align}
	\vec{\grad}\cdot\vec{v} = \frac{1}{J}\pdv{(JA^i)}{u^i} \label{eq:MetricDivergence}
\end{align}

and further the divergence of parallel vector fields $S_\parallel\vec{b}$ comes in handy: 
\begin{align}
	\vec{\grad}\cdot \left(S_\parallel\vec{b}\right) = \frac{1}{J}\left[\pdv{\left(J S_\parallel b^\theta\right)}{u^\theta} + \pdv{\left(J S_\parallel b^\varphi\right)}{u^\varphi}\right] \label{eq:MetricDivergenceParallel}
\end{align}

The gradient and divergence operators can be combined o define a diffusion operator. The parallel Laplacian with some diffusion coefficients $D$ reads in metric terms: 
\begin{align}
	\label{eq:MetricParallelDiffusion}
	\vec{\grad}\cdot\left[D\left(\grad_\parallel S\right)\vec{b}\right] = \frac{1}{J}\left[\pdv{\left(JD \left(\pdv{S}{u^\theta}b^\theta + \pdv{S}{u^\varphi}b^\varphi\right) b^\theta\right)}{u^\theta} + \pdv{\left(JD\left(\pdv{S}{u^\theta}b^\theta + \pdv{S}{u^\varphi}b^\varphi\right) b^\varphi\right)}{u^\varphi}\right]
\end{align}

Similarly, a perpendicular diffusion operator can be defined: 
\begin{align}
	\vec{\grad}\cdot\left[D\vec{\grad}_\perp S\right] =& \vec{\grad}\cdot\left[D\left(\vec{\grad} S - \left(\grad_\parallel S\right)\vec{b}\right)\right] \nonumber \\
	=& \frac{1}{J}\pdv{}{u^i}\left[JD\left(g^{ij}\pdv{S}{u^j} - \left(\pdv{S}{u^\theta}b^\theta + \pdv{S}{u^\varphi}b^\varphi\right)b^i\right)\right] \nonumber \\
	=& \frac{1}{J}\left(\pdv{}{u^\psi}\left[JD\left(g^{\psi\psi}\pdv{S}{u^\psi} + g^{\psi\varphi}\pdv{S}{u^\theta} + g^{\psi\theta}\pdv{S}{u^\varphi}\right)\right]      \right.\nonumber \\
	&+ \pdv{}{u^\theta}\left[JD\left(g^{\theta\psi}\pdv{S}{u^\psi} + \left(g^{\theta\theta} - b^\theta b^\theta\right)\pdv{S}{u^\theta} + \left(g^{\theta\varphi} - b^\varphi b^\theta\right)\pdv{S}{u^\varphi}\right)\right] \nonumber \\
	&+ \left. \pdv{}{u^\varphi}\left[JD\left(g^{\varphi\psi}\pdv{S}{u^\psi} + \left(g^{\varphi\varphi} - b^\theta b^\varphi\right)\pdv{S}{u^\theta} + \left(g^{\varphi\theta} - b^\varphi b^\varphi\right)\pdv{S}{u^\varphi}\right)\right]\right) \label{eq:MetricPerpendicularDiffusion}
\end{align}


\section{Staggered operators}
